\chapter{Lab Safety}\label{lab-safety}

The laboratory classes are hands-on. Some classes require the use of
hazardous chemicals and materials. Safety in the classroom is the \#1
priority for students and faculty. To ensure a safe science laboratory,
a list of rules must be followed at all times.

Please watch the \href{https//youtu.be/NxcsyTv7stQ}{RCC safety video}

\BeginKnitrBlock{rmdimportant}
\textbf{Prior to your participation in the science lab course, you must
read this safety document and sign and return the acknowledgment and
agreement page.}
\EndKnitrBlock{rmdimportant}

\section{General rules}\label{general-rules}

\begin{enumerate}
\def\labelenumi{\arabic{enumi}.}
\tightlist
\item
  NO FOOD, BEVERAGES, GUM, in the labs. Cell phone usage is also
  prohibited in the lab.
\item
  Conduct yourself in a responsible manner at all times in the lab.
  Horseplay, pranks and practical jokes are prohibited and will not be
  tolerated. If you participate in inappropriate behavior the INSTRUCTOR
  HAS THE RIGHT TO ASK YOU TO LEAVE THE lab.
\item
  Students cannot be in the lab without an instructor present.
\item
  Read all lab procedures, precautions, and equipment instructions
  thoroughly before each lab. Follow all written and verbal instructions
  carefully. Perform only those experiments authorized by the
  instructor. If during the lab you don't understand, stop and ask the
  instructor before proceeding. Never do anything in the lab that is
  outside of your instructors directions or that is not in your lab
  procedure.
\item
  Do not begin lab activities; touch any chemicals or equipment until
  you are instructed to do so.
\item
  Work areas should be kept organized and clean at all times. Only
  necessary items (lab notebook, worksheets, etc.) should be on your
  workbench. Backpacks and purses must be stored under the benches or
  against the walls. CLEAN ALL OF YOUR WORK SURFACES AND EQUIPMENT AT
  THE END OF THE EXPERIMENT. Safely dispose of waste in its proper
  container and place glassware in the grey bins by the sink. DO NOT
  STACK GLASSWARE. If the bin is full ask the instructor for another
  bin.
\item
  Keep aisles clear. Push lab stools under the lab benches when not in
  use.
\item
  Know where the safety equipment is and how to use it. This includes
  the first aid kit, eyewash station, safety shower, fire extinguisher,
  and fire blanket. Know the location of the fire alarm, and emergency
  phone. In the event of a fire drill during lab time containers must be
  closed, gas valves off, fume hood, and all electrical equipment must
  be turned off.
\item
  NEVER DISPOSE OF ANYTHING IN THE SINK. All materials are to be
  disposed of in the proper hazardous waste containers with the
  assistance of the instructor. All waste containers must be closed and
  placed inside a secondary containment bin.
\item
  As classes in these labs use toxic chemicals, keep your hands away
  from your face, eyes and mouth while working in the lab. Always wash
  your hands thoroughly with warm water and soap before leaving the lab
  to prevent injury or illness. This is part of proper lab procedure.
\item
  Students are not permitted in the prep room areas (between the lab
  rooms).
\item
  Handle all living organisms used for lab experiments in a respectful,
  humane manner.
\item
  Microscopes must be properly cleaned, the electrical cords properly
  wrapped, and returned to their places with their protective covers.
\end{enumerate}

Disposal of all hazardous waste is ONLY to be handled by the instructor
and in a manner consistent with federal, state and local hazardous waste
disposal regulations. Organic solvents are never to be disposed of down
the sink; receptacles will be provided as needed for their collection.
All hazardous chemical substances must be placed in the appropriate type
of container and labeled with chemical, name and date, sealed and placed
upright in a gray plastic bin.

\section{Class dissections}\label{class-dissections}

\begin{enumerate}
\def\labelenumi{\arabic{enumi}.}
\setcounter{enumi}{13}
\tightlist
\item
  Preserved biological specimens should be treated with respect and
  disposed of in a clear plastic bag and placed inside the hazardous
  waste drum located in the classroom. This container must be sealed at
  the end of each dissection.
\item
  When using sharp objects always carry the tips pointed down and away
  from you. When dissecting, cut away from your body. Grasp the
  instrument only by the handle. Never try to catch falling sharp
  instruments or glassware. When you are finished dissecting, wash and
  dry your instruments and dissecting pan before storing them in the
  proper location. Do not leave any instruments in the sink.
\end{enumerate}

\section{Clothing}\label{clothing}

\begin{enumerate}
\def\labelenumi{\arabic{enumi}.}
\setcounter{enumi}{15}
\tightlist
\item
  Students must wear lab goggles when using chemicals, or using heat. NO
  EXCEPTIONS! Lab coats are mandatory for Anatomy and Physiology,
  Biology, Microbiology, Chemistry, and Biotechnology labs, except for
  when lecturing and working on dry lab (for example, looking at models
  or prepared slides) activities.
\item
  Gloves should be worn when handling solutions, solids, specimens, etc.
\item
  Proper dress should always be observed in the lab. Long hair must be
  tied back. Loose or baggy clothing (especially sleeves), dangling
  jewelry, hats, shorts, short skirts, bare mid riffs, high heels,
  sleeveless shirts, and open toed or open heeled shoes and sandals are
  prohibited in the lab. Failure to comply may result in expulsion from
  class.
\end{enumerate}

\section{Handling chemicals}\label{handling-chemicals}

\begin{enumerate}
\def\labelenumi{\arabic{enumi}.}
\setcounter{enumi}{17}
\tightlist
\item
  Always work in a well-ventilated area. Use the fume hood when working
  with volatile substances or poisonous vapors, or any chemical with an
  odor.
\item
  Never smell a chemical by sniffing. Use your hand to wave the chemical
  towards your nose.
\item
  DO NOT TASTE, TOUCH or smell anything unless instructed to do so. You
  should wear a lab apron and gloves at all times. Your instructor will
  tell you the proper gloves to wear depending on the chemical being
  used.
\item
  CHECK EACH LABEL TWICE before removing any of its contents. Take only
  what is needed of each chemical. NEVER return unused chemicals to
  their original container; put it in the waste container.
\item
  Do not use your fingers to transfer solid chemicals. Use a scoop or
  spatula.
\item
  Use a rubber bulb, pipette, or pi-pump when transferring liquid
  chemicals. NEVER USE YOUR MOUTH TO PIPETTE!
\item
  When transferring reagents hold the containers away from your body
  while working on the bench.
\item
  Acids must be handled with extreme care. You will be shown the proper
  method for diluting strong acids. ALWAYS ADD ACID TO WATER, swirl or
  invert the solution and be careful of the heat produced particularly
  with sulfuric acid.
\item
  Handle hazardous liquids over a pan to contain spills.
\item
  Never handle flammable liquids anywhere near an open flame or heat
  source.
\item
  Be careful when transporting chemicals across the lab. Hold securely
  and walk carefully.
\item
  NEVER POUR CHEMICALS INTO SINK. Waste should be disposed of in the
  proper hazardous waste container provided.
\end{enumerate}

\section{Glassware}\label{glassware}

\begin{enumerate}
\def\labelenumi{\arabic{enumi}.}
\setcounter{enumi}{29}
\tightlist
\item
  Never handle broken glass with bare hands. Use a brush and a dustpan
  to clean up broken glass. Place uncontaminated broken glass in the
  white and blue broken glass receptacles. Contaminated trash goes in
  the biohazard bin.
\item
  Fill wash bottles only with distilled water and use only as intended,
  e.g.; rinsing glassware, adding water to a container.
\item
  Never use chipped, cracked or dirty glassware to avoid shattering.
\item
  Never immerse hot glassware in cold water. It may shatter.
\item
  Never place dirty glassware with the clean glassware. All dirty
  glassware should be placed in gray wash bins. DO NOT STACK DIRTY
  GLASSWARE IN BINS.
\end{enumerate}

\section{Heating substances}\label{heating-substances}

\begin{enumerate}
\def\labelenumi{\arabic{enumi}.}
\setcounter{enumi}{34}
\tightlist
\item
  Exercise extreme caution when using a gas burner. Be careful to keep
  hair, loose clothing and hands away from flames at all times. Wear
  safety goggles. Do not put any substances into the flame unless
  specifically instructed to do so. Never reach over an exposed flame.
  The instructor will provide a demonstration of the proper way to
  operate a Bunsen burner. Never leave a lighted burner or hot plate
  unattended. Always turn the burner or hot plate off when not in use.
\item
  Do not point the open end of a test tube being heated at yourself or
  anyone else. Never look into a container that's being heated.
\item
  Heated metals and glass remain hot for a very long time. They should
  be set aside to cool and picked up with caution. Use tongs or heat
  protective gloves.
\end{enumerate}

\section{Handling microbiology
materials}\label{handling-microbiology-materials}

\begin{enumerate}
\def\labelenumi{\arabic{enumi}.}
\setcounter{enumi}{37}
\tightlist
\item
  Please be aware that micro labs include work with pathogenic
  organisms. Be alert. Conduct yourself in a responsible manner at all
  times.
\item
  If you spill anything notify your instructor immediately. There are
  special procedures to be followed for spills containing
  microorganisms.
\item
  A lab coat must be worn during lab activities. Lab coats/aprons may
  never leave the lab. If you must leave the lab during a class then
  your lab coat must be removed.
\item
  Gloves must be worn at all times when working with bacteria. Gloves
  need to be disposed of in the biohazard waste container.
\item
  All contaminated waste must be disposed of in the biohazard container.
  Do not overfill biohazard containers.
\item
  You must spray down your lab bench with Lysol after each lab. Do not
  wipe with paper towels. The bench surface must remain wet for at least
  five minutes for the Lysol to destroy any micro organisms.
\item
  Wash your hands thoroughly before and after each lab as well as before
  you leave the lab for any reason.
\item
  Dispose of contaminated broken glass in the biohazard bin. Please wrap
  the broken glass in paper towels before disposal so that the broken
  glass doesn't cut through the bag. Dispose of uncontaminated glass in
  the white and blue cardboard glass boxes.
\end{enumerate}

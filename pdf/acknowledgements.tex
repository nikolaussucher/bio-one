\chapter*{Acknowledgements}\label{acknowledgements}
\addcontentsline{toc}{chapter}{Acknowledgements}

At \href{http://www.rcc.mass.edu}{RCC}, introductory (general) biology
is split into two courses (SCI103, Biology I, and SCI104, Biology II)
which are taught over two semesters. The two courses were originally
developed in the 1970s, by Prof.~Georgia Whitman essentially
representing botany (Biology I) and zoology (Biology II). Some 20 years
ago, Prof.~Kyrsis Rodriguez who served as the biology course coordinator
until her retirement reduced the botany content and introduced molecular
aspects of biology.

In 2015, the Massachusetts Department of Higher Education with its
\href{http://www.mass.edu/masstransfer/}{MassTransfer Pathways}
initiative prompted a statewide critical evaluation of the structure and
content of foundational courses, such as introductory biology, in order
to make these courses eligible for transfer across the Commonwealth
amongst public higher education institutions. In 2016, we have been
fortunate to receive a large grant from the
\href{http://www.masslifesciences.com}{Massachusetts Life Sciences
Center} to upgrade and modernize our science labs. We have taken this
opportunity to re-structure our courses along the lines of current
practice of introductory biology teaching such that the Biology I course
introduces the molecular and cellular basis of life, while the Biology
II course is focused on the basics of organismal biology. Overall the
proposed changes will expose our students to modernized (up-to-date)
concepts of biology and thus make them more competitive.

This laboratory manual was inspired by
``\href{https://www.amazon.com/Encounters-Life-General-Biology-Laboratory/dp/0895826852}{Encounters
with Life}'' by Larry J. Scott and Hans F. E. Wachtmeister published by
the Morton Publishing Company in Englewood CO. We used that manual at
RCC for many years. Progress in the biological sciences is, however,
relentless and so over the years, our requirements changed, and we felt
that it was best to put together a new manual specifically tailored to
our needs.

The creation of this manual was greatly facilitated by and owes a major
debt to Wikipedia and its large number of voluntary contributors. I very
liberally copied from many Wikipedia pages and then remixed, edited,
adapted and added text. With your continued support and help this manual
can only get better over time. I urge you to email me with your
criticisms and suggestions at
\href{mailto:nsucher@rcc.mass.edu}{\nolinkurl{nsucher@rcc.mass.edu}}.
This manual is as open educational resource licensed  under
\href{https://creativecommons.org/licenses/by-sa/3.0/deed.en}{Creative
Commons Attribution-Share Alike 3.0 Unported} United States License for
others to do as I did and improve and adapt to specific requirements.

I wish to thank Dr.~Hillel Sims, Dean of STEM, for the laboratory safety
and microscope videos. Thanks to Prof.~Bruce Brender for suggesting the
catalase experiments, which are based on the experiments in the Lab
Manual for Biology by Sylvia Mader published by McGraw-Hill. I wish to
thank Andrea Thomason and Dr.~Maria Carles for the strawberry DNA
extraction protocol. Thanks also go to Daveen Blythe, Director of
Science Laboratories at RCC, for working hard setting up the new
experiments. Last but not least, I want to thank all lab technicians, my
teaching colleagues and our students who work together to translate mere
words into an exciting laboratory experience for all of us at RCC.
